\documentclass[authoryearcitations]{UoYCSproject}
\author{Jeremy L. Jacob}
\title{A guide to type-setting project reports in \LaTeXe\ with the
  \textsf{UoYCSproject} class}
\date{Version 2.16, 2010-April-02}
\supervisor{Jeremy L. Jacob}
\MIP
\wordcount{8832}

\includes{Appendices \ref{cha:usefulpackages}, \ref{cha:gotchas} and
  \ref{cha:deptfac}}

\excludes{\autoref{cha:quoteex}}

\abstract{ \LaTeXe\ is a document markup and processing system built
  upon Donald Knuth's type-setting system, \TeX.
   
  \lstinline|UoYCSproject| is a \LaTeXe\ class for producing reports
  describing projects taken as part of a taught course in the
  Department of Computer Science at the University of York.  (It is
  not designed for research degree reports.)
  
  A brief introduction to \LaTeXe\ is given.  The
  \lstinline|UoYCSproject| class is described.
  
  This document itself is (inappropriately) an example of the use of
  the class \lstinline|UoYCSproject|.
}

\dedication{To all students everywhere}

\acknowledgements{
  I would like to thank my goldfish for all the help it gave me
  writing this document.
 
  As usual, my boss was an inspiring source of sagacious advice.
}

% More definitions & declarations in example.ldf

\begin{document}
\maketitle
\listoffigures
\listoftables
\renewcommand*{\lstlistlistingname}{List of Listings}
\lstlistoflistings

\cleardoublepage
\part{Preliminaries}
\label{sec:start}
\thispagestyle{empty}\cleardoublepage

\chapter{Introduction}
\label{cha:Introduction}

In each taught course, undergraduate or postgraduate, there is a
compulsory large project.\footnote{Except for the three teaching-year
  joint degrees with mathematics, where a computer science project is
  optional.}  By far the largest component of the assessment of the
project is a written report.  There are various appropriate
technologies for producing reports.  Among these is Lamport's \LaTeX\ 
\citep{Lamport1994}.

This user guide describes a \LaTeXe\ class, \lstinline|UoYCSproject|,
to help in the type-setting of project reports; it is
(inappropriately) written using that document class.  The division
into parts, chapters and so on is too heavy for a brief introduction
and user guide, but appropriate for a project report.  The source code
for this document is available through the
\href{http://www-course.cs.york.ac.uk/csw/}{CSW web site}; you are
welcome to use it as a template.

\section{What is \LaTeX?}
\label{sec:whatislatex}

\LaTeX, or more strictly, \LaTeXe, is a notation for describing
document structure (much as HTML or XML applications)
\citep{Lamport1994}.  It is very different from \textsc{wysiwyg},
which has been characterised as ``What you see is all you've
got''\footnote{\citet[p7, Footnote~1]{Lamport1994} says that ``Brian
  Reid attributed this phrase to himself and/or Brian Kernighan''.}

\LaTeXe\ is built on top of Donald Knuth's \TeX\ \citep{Knuth1984}.
\TeX\ is a notation for describing type-set pages \emph{plus} a macro
language.  \LaTeXe\ is a collection of \TeX\ macros that allows for
extensions and modifications using the class and package mechanisms.
Thus a \LaTeXe\ description of a document can be turned into print by
processing it with a suitable program.

Output is available as the original Device Independent (DVI) format
(by using \lstinline|latex| to process the document), PostScript (by
converting from DVI) or PDF (by using \lstinline|pdflatex| to process
the document).

\TeX\ itself was developed by Donald Knuth for type-setting his books,
particularly his multi-part work on algorithms
\citep{Knuth1997,Knuth1998a,Knuth1998b}; take a look at them to see
what is possible.  He also developed a font design program to
accompany \TeX, \MF.\footnote{An illustration of how Donald Knuth's
  mind works.  The current version of \TeX\ is 3.141592; the next
  version, should there be one, will be numbered 3.1415926, and the
  one after that 3.14159265.  On his death the source code is to be
  amended to print out `Version \textbackslash Pi', and no further
  changes will be allowed.  Similarly, \MF\ version numbers are
  converging on \begin{math}e\end{math}; currently it is
  Version~2.71828.}

\section{Advantages of \TeX}
\label{sec:advantagesoftex}

\TeX\ has a very sophisticated text type-setting algorithm; its
implementation is proved optimal (Donald Knuth did more or less found
the theory of algorithms).  The PDF\TeX\ engine extends the algorithm
to include hanging punctuation, for even better results.  (See
\href{http://www.tug.org/texshowcase/}{the \TeX\ showcase} for several
examples; it lives at \url{http://www.tug.org/texshowcase/}.)

\TeX\ has a very sophisticated mathematics type-setting algorithm.

\TeX\ also has a Turing-equivalent macro language so that you can
program substructures in your document.

\section{Advantages of \LaTeXe}
\label{sec:advantagesoflatex}

\LaTeXe\ provides a pre-defined set of document structures (using the
\TeX\ macro language), and hooks for integrating further structures.

\LaTeXe\ simplifies the task of writing \TeX\ macros (unless you need
something very sophisticated).

\section{Advantages of a programmable mark-up language}
\label{sec:advantagesofprogrammable}

I consider the ability to write definitions the greatest advantage of
\TeX-like systems.

Such a facility enables its users to design a collection of macros
that reflect the abstract syntax of important structures in the
document (later we will see an example of part of a collection of
macros for describing cryptographic protocols).  Just doing this will
help you ask the right questions about your project, even if you end
up using some other document processing system.  The fact that
\LaTeXe\ also lets you associate type-setting commands with each
element of the abstract syntax is an added bonus, and one that gives
you consistent type-setting across the document, and between
documents.

\cleardoublepage
\chapter{Useful references}
\label{cha:usefulrefs}


\section{Books}
\label{sec:books}

\begin{description}
\item[\citet{Lamport1994}] The original source.  It has a reasonable
  reference manual, but can be terse.  It does not cover package and
  class writing, nor does it cover more than a handful of useful
  packages.  It does describe the \BibTeX\ and index making programs.
\item[\citet{KopkaDaly1999}] A comprehensive reference; it covers
  everything except the many add-on packages.  Most people use this as
  their primary reference.
\item[\citet{MittelbachGoossensBraamsCarlisleRowley2004}] A guide to
  many of the most useful add-on packages and classes.
\item[\citet{GoosensRahtzMittelbach1997}] A slightly dated guide to
  packages for graphics.
\item[\citet{GoosensRahtzGurariMooreSutor1999}] A slightly dated guide
  to packages for adding hyperlinks, and producing PDF and HTML from
  \LaTeXe.
\end{description}


\section{Papers}
\label{sec:papers}

There are many papers describing \LaTeXe\ and its associated packages.
They are available on-line, usually through the Comprehensive \TeX\
Archive Network.\footnote{\CTAN, \url{http://www.ctan.org/.}}.  They
are usually also available on the \TeX~Live
distribution,\footnote{\url{http://www.tug.org/texlive/}} which the
department uses.\footnote{Departmental Linux users should look under
  \url{file:///usr/local/pkg/} for the current \TeX~Live distribution,
  and under that for the various \lstinline|doc| directories;
  documentation is usually in \lstinline|pdf| or \lstinline|dvi|
  files.}

The useful \emph{general} papers are:
\begin{description}
\item[The Not So Short Introduction to \LaTeXe\ 
  \citep{OetikerPartlHynaSchlegl2002}]\
  
  Available in several languages.
  
  Be warned that this paper describes the standard classes.  There are
  a few differences in the class options and declarations bewteen the
  standard classes and \lstinline|UoYCSProject|.
\item[Math mode \citep{Voss2007}] A detailed explanation of
  typesetting mathematics in \LaTeXe.
\item[The Comprehensive \LaTeX\ Symbol List \citep{Pakin2005}] An
  enormous list of symbols and how to make them.\footnote{An
    experimental web application for finding symbols can be found at
    \url{http://detexify.kirelabs.org/}.}
\item[Packages in the `graphics' bundle \citep{Carlisle1999}] A bit
  out of date (it does not describe PDF extensions), but a useful
  introduction.
\item[Hypertext marks in \LaTeX\ \citep{RahtzOberdiek2003}] Access to
  hypertext features via the \lstinline|hyperref| package.

  Most of the effects happen automatically on loading the package.

  It works best in combination with the \lstinline{hypcap} package.

  \lstinline|UoYCSproject| loads these packages for you, and sets some
  of the manual things to sensible values.
\item[The \KOMAScript\ bundle \citep{KohmMorawski2003}]\

  \lstinline|UoYCSproject| is based on the \KOMAScript\ 
  \lstinline|scrreprt| class.  The manual will tell you about several
  extra facilities available to you (but you should not change layout,
  and such things).
\end{description}

\section{Web resources}
\label{sec:webresources}

\newcommand{\wwwr}[2]{\item[\href{#1}{#2} \textless\url{#1}\textgreater]}
\begin{description}
  \wwwr{http://www.ctan.org/}{The Comprehensive \TeX\ Archive Network}\
  
  What it says on the label.  Almost everything you need in the way of
  \TeX\ and friends can be found here.  Also known as \CTAN.

\wwwr{http://www.tug.org/}{The \TeX\ Users Group (TUG)} A useful web
  site.
  
  TUG members get the \TeX~Live distribution as part of their
  subscription.

\wwwr{http://faq.tug.org/}{\TeX~FAQ}\

  An extremely useful first port of call for solving common problems,
  hosted by TUG.

\wwwr{http://tug.org/pracjourn/}{The Prac\TeX\ Journal}\

  An on-line journal of \TeX\ practice, including a Q\&A section.

\wwwr{http://www.latex-project.org/}{The \LaTeX\ Project}\

  The centre of the \LaTeX\ project.
  
\wwwr{news:comp.text.tex}{The \TeX\ newsgroup}
  
  If asked politely, questions not in the FAQ or standard sources of
  documentation will usually be answered by gurus.  \emph{Minimal}
  examples of problems, together with the versions of \TeX, \LaTeX\ 
  and all classes and packages used in the example must be given.

\wwwr{http://research.silmaril.ie/latex/}{Peter Flynn's `Formatting
  information'}

  \ \\An on-line \LaTeXe{} manual.

\wwwr{http://dartar.free.fr/w/?wakka=latex}{The beauty of \LaTeX{}}

  A page describing typographic advantages of \TeX-based systems over
  common competitors.

\wwwr{http://en.wikibooks.org/wiki/LaTeX}{A wiki for \LaTeXe{}}

  A relatively new resource; as good or as bad as a Wiki can be.

\wwwr{http://www.tex.ac.uk/tex-archive/info/visualFAQ/visualFAQ.pdf}{A Visual FAQ for \LaTeXe{}}

  The associated README file for this resource says:
  \begin{quote}\small
    Having trouble finding the answer to a LaTeX question?  The Visual
    LaTeX FAQ is an innovative new search interface that presents over
    a hundred typeset samples of frequently requested document
    formatting.  Simply click on a hyperlinked piece of text and the
    Visual LaTeX FAQ will send your Web browser to the appropriate
    page in the UK TeX FAQ.
  \end{quote}

\wwwr{http://www.mathtran.org/toys/jfine/editor2.html}{MathTran
  instant preview}

A web-based application to let you try out small pieces of \TeX\
(\emph{not} \LaTeXe) source code (especially mathematical source code)
to see what the type-set version looks like.
\end{description}

\cleardoublepage
\chapter{The \LaTeX\ edit-process cycle}
\label{cha:editprocess}

The standard books on \LaTeX\ describe the process by which you turn
your document description into ink.  Most describe this process using
\lstinline|latex|, which produces DVI format, and a DVI viewer, such
as \lstinline|xdvi|.

Since these books were written it has become more convenient to use
\lstinline|pdflatex|, which produces PDF format, and a PDF viewer such
as \lstinline|xpdf| or \lstinline|acroread| (\lstinline|xpdf| is
slightly more convenient than \lstinline|acroread|, although it does
not support all the features that \lstinline|acroread| does, nor does
it have as good rendering).

\LaTeXe\ source may be created using any editor.  Several editors have
support for \TeX\ and \LaTeXe, including managing the edit-create
cycle.  I like \lstinline|emacs| with the AUC\TeX\ enhancements to the
\TeX\ modes; Windows users often use \lstinline|WinEDT|.

The perfect edit-process cycle goes like this:
\begin{enumerate}
\item Create a \LaTeXe{} source file, and any others needed, such as a
  \BibTeX{} file, figures, and so on.
\item Run \lstinline|pdflatex|.  (This creates PDF output with
  place-holders for missing information and auxiliary files with
  information about the table of contents, cross references, name of
  file(s) containing the bibliographic database, and so on.)
\item Run \BibTeX.  (This creates a file containing the references.)
\item Run \lstinline|pdflatex|.  (This recreates PDF output with
  place-holders for missing information and auxiliary files with
  information about the table of contents, cross references, name of
  file(s) containing the bibliographic database, and so on, but this
  time also with bibliographic citations.)
\item Run \lstinline|pdflatex|. (This will create PDF output which is
  complete.)
\end{enumerate}
Imperfections in this cycle creep in when you make errors in the
files, add new citations, and so on.  Further recompilation is
necessary; rerunning \BibTeX{} is only necessary if new citations are
inserted or if an entry in the bibliographic database changes.

Tools such as AUC\TeX{}/ \lstinline|emacs| and \lstinline|WinEDT| can
manage the process for you.

A brief guide to using \LaTeXe{} on (some of) the department's systems
is given in \autoref{cha:deptfac}.


\setpartpreamble{%
  \vspace*{4ex}
  
  In this part of the document I briefly review some of the main
  concepts of \LaTeXe\ documents.
  
  This is \emph{not} a comprehensive guide to \LaTeXe, but a list of
  useful concepts, together with a few hints and tips. Consult the
  main references for full details.

}

\cleardoublepage
\part{Concepts of \LaTeXe}
\label{sec:middle}
\thispagestyle{empty}\cleardoublepage

\chapter{The anatomy of a \LaTeXe\ source file}
\label{cha:anatomy}

The layout of a normal \LaTeXe\ document description is given in
\autoref{lst:anatomy}.
\begin{lstlisting}[%
  caption={The anatomy of a \LaTeXe\ file},%
  label=lst:anatomy,%
  float,%
  morekeywords={document,maketitle},%
  numbers=left%
  ]
  \documentclass[class options]{class name}
    preamble (definitions and declarations)
  \begin{document} % this is a comment, from the `%' to the `<cr>'.
    \maketitle % to generate the title information
    body 
  \end{document}
\end{lstlisting}

On Line~1 is the \emph{document class declaration}.  This declares the
class to which the document belongs, as the mandatory parameter to the
\lstinline|documentclass| command; mandatory parameters appear in
curly braces.  Most classes have optional parameters; these are passed
in the square brackets.  Optional parameters for most commands appear
in an unusual position when they do appear: between the command name
and the mandatory parameters.

Next (represented by Line~2 of \autoref{lst:anatomy}) is the preamble,
which contains further definitions and declarations for the document.
This can stretch over many lines.  Usually there is a great deal of
freedom about what can appear here; the class \lstinline|UoYCSproject|
is very restricted, and introduces a separate mechanism for private
declarations (see \autoref{cha:UoYCSproject}).

The document body is delimited by the markers on Line~3 and Line~5.
In between goes the document structured into (optional parts,)
chapters, sections, subsections and so on, represented here by Line~4.

\cleardoublepage
\chapter{Definitions and Declarations}
\label{cha:dand}


\section{Declarations}
\label{sec:declarations}

Declarations are easiest to deal with, so we describe them first.
There are two kinds: individual items and packages of related items.


\subsection{Individual declarations}
\label{sec:individualdeclarations}

Most classes and packages allow or mandate features of the document to
be set by declaration.  The syntax is a command that names the
declaration and a parameter that gives the value.  For example, all
classes that have a title have a declaration to set it: see
\autoref{lst:title}.
\begin{lstlisting}[float,caption={Declaring title matter},label={lst:title}]
  \title{text}
  \author{name 1 \and name 2 \and name 3}
  \date{text}
\end{lstlisting}
Along with the title usually goes an author (or authors) and an
optional date (if not given, the date defaults to the date the file is
processed); again see \autoref{lst:title}.

Some classes, such as \lstinline|UoYCSproject|, have a larger
collection of declarations.  (The declarations made available by
\lstinline|UoYCSproject| are given in
\autoref{tab:UoYCSpdeclarations}.)


\subsection{Package loading}
\label{sec:packages}

Often a document contains structures that are orthogonal to the
document structure.  A common example in computer science projects is
a code listing.  A \emph{package} is a collection of definitions that
supports marking up the structures.  The \lstinline|listings| package
is recommended for marking up code fragments (that package has been
used for the fragments of \LaTeXe\ code in this document).

Note that \lstinline{UoYCSproject} provides a different, non-standard,
place for you to load packages.  See \autoref{sec:uoycsp:diy}.

Packages are loaded with the command
\lstinline|\usepackage{package name}|.
An example is given in \autoref{lst:package}
\begin{lstlisting}[
  float,
  caption={Loading a package},
  label={lst:package}
  ]
  \usepackage{listings} % for pretty printed code listings
\end{lstlisting}
They often have large numbers of optional parameters, and associated
declarations to control their behaviour.  The description should be
given in the package documentation.

There are very many packages available; see examples given in
\autoref{cha:usefulrefs} and \autoref{cha:usefulpackages} and the web
site \url{http://www.tex.ac.uk/tex-archive/help/Catalogue/}.

\section{Definitions}
\label{sec:definitions}

It is the ability to make definitions that gives \LaTeXe\ its real
power.  Commands can be defined to express the logical structure of
the concepts in your project, and these can be separated from their
mark-up.

Note that \lstinline{UoYCSproject} provides a different, non-standard,
place for you to load packages.  See \autoref{sec:uoycsp:diy}.

There are two kinds of definitions: commands and environments.


\subsection{Commands}
\label{sec:commands}

New commands are declared with the \lstinline|\newcommand| or
\lstinline|\newcommand*| command.

The simplest use is when you have a long phrase that you need to type
regularly, and you wish to save yourself some keystrokes and/or ensure
consistency between occurrences.  An example is given in
\autoref{lst:newcommanduoy}.
\begin{lstlisting}[%
  float,%
  caption={A new command without parameters},%
  label={lst:newcommanduoy}]
  \newcommand*{\uoy}{The University of York}
\end{lstlisting}
Anywhere that `\lstinline|\uoy|' occurs in the scope of the definition
the text `The University of York' is substituted.  The definition is
designed to be used in a \emph{text mode} rather than a \emph{math
  mode} (see \autoref{sec:modes}).\footnote{If called in a math mode
  the result is `$The University of York$'!}

Commands can also have parameters; and this is where the two forms of
definition differ from each other.  \lstinline|\newcommand*| defines a
command whose parameters may \emph{not} include paragraph breaks
(`short' parameters in \TeX\ parlance); \lstinline|\newcommand|
defines a command whose parameters \emph{may} include paragraph breaks
(`long' parameters in \TeX\ parlance).  The `starred' form is almost
always the appropriate one.

As an example, \autoref{lst:newcommandmsg} shows how to define a
command, called \lstinline|\msg|, to typeset a message in a protocol;
the message has three parts: sender, intended recipient and body.  The
command is to be used in a math mode, and later we define an
environment for whole protocols.

The optional parameter following the name of the command being defined
is the number of parameters (maximum: 9) that the command has; these
parameters are called \lstinline|#1|, \lstinline|#2| and
\lstinline|#3|.
\begin{lstlisting}[%
  float,%
  caption={A new command with parameters},%
  label={lst:newcommandmsg}%
  ]
  \newcommand*{\msg}[3]{#1\rightarrow#2:#3}
\end{lstlisting}
As an example of its use, `$A\rightarrow B:M,K(A,B,N)$' may be typeset
by the call `\lstinline|\msg{A}{B}{M,K(A,B,N)}|'.

Now suppose that you wish to change the printed format of a message
everywhere in the document: all you need to do is to modify the body
of the definition.  Alternative definitions are given in
\autoref{lst:newcommandmsg2} and \autoref{lst:newcommandmsg3}.
\begin{lstlisting}[%
  float,%
  caption={A second new command with parameters},%
  label={lst:newcommandmsg2}%
  ]
  \newcommand*{\msg}[3]{%
    #2\Longleftarrow\left[#3\right]\Longleftarrow#1}
\end{lstlisting}
\begin{lstlisting}[%
  float,%
  caption={A third new command with parameters},%
  label={lst:newcommandmsg3}%
  ]
  \newcommand*{\msg}[3]{
    \begin{array}{@{}c@{}}
         #1
      \\ \bigtriangledown
      \\ #3
      \\ \bigtriangledown
      \\ #2
    \end{array}
  }
\end{lstlisting}
The second definition of \lstinline|\msg| typesets the call
`\lstinline|\msg{A}{B}{M,K(A,B,N)}|' as
\begin{displaymath}
B\Longleftarrow\left[M,K(A,B,N)\right]\Longleftarrow A
\end{displaymath}
while the third typesets it as
\begin{displaymath}
\begin{array}{@{}c@{}}A\\\bigtriangledown\\M,K(A,B,N)\\\bigtriangledown\\B\end{array}
\end{displaymath}

The \lstinline|\newcommand| commands will report an error if the
command name is already defined (possibly in the environment).  You
can \emph{redefine} a command by using \lstinline|\renewcommand| and
\lstinline|\renewcommand*|.  There are other subtle variations on
command definition, including the ability define commands with one
optional parameter.  For these, see a standard book on \LaTeXe, such
as those listed in \autoref{sec:books}.\footnote{The underlying \TeX\
  definition mechanism is extremely powerful, allowing a much greater
  flexibility in the syntax of introduced commands.  See
  \citet{Knuth1984}.}

\subsection{Environments}
\label{sec:environments}

An \emph{environment} is used to group together a structure.  An
instance of environment \lstinline|e| begins with
\lstinline|\begin{e}| and ends with \lstinline|\end{e}|.

For example, there are predefined environments for various types of
lists (see \autoref{lst:listex}), for quotations (see
\autoref{lst:quoteex}; note the use of a comment to break a long word
and hide the new-line character and the use of \lstinline|\-| to state
additional places where hyphenation is allowed; this example is
typeset in \autoref{cha:quoteex}), and for arranging formul\ae\ (the
\lstinline|array| environment in \autoref{lst:newcommandmsg3}).
\begin{lstlisting}[
  float,
  caption={An example of a bulleted list},
  label={lst:listex}
  ]
  \begin{itemize}
  \item The first bullet point.
  \item And now the second.
  \item Followed by a third.
  \end{itemize}
\end{lstlisting}
\lstinputlisting[
  float,
  caption={An example of quotations},
  label={lst:quoteex},
  basicstyle=\sffamily\small
  ]{joyce.tex}

Declaring an environment is very like making a definition, except that
now we have to give code for the start and the end of the environment.
In \autoref{lst:mqex} I show how to define an environment that behaves
like the \lstinline|quote| environment, except that the font used is
small and italic.  It also takes one parameter, a citation label,
which causes the citation to be printed at the bottom right hand side
of the quote, preceded by an em-dash.
\begin{lstlisting}[
  float,
  caption={A new quote environment},
  label={lst:mqex}
  ]
  \newenvironment*{mq}[1]
    {\begin{quote}\small\itshape\newcommand*{\cl}{#1}}% begin code
    {\par\hspace*{\fill}---\citep{\cl}\end{quote}}% end code
\end{lstlisting}
The command used to define the environment is
\lstinline|\newenvironment*| (there is also an un-starred version, as
well as `renew' versions).  The environment's name is \lstinline|mq|.
It has one parameter.  Next comes the code to be executed at the start
of the environment: begin a quote environment, set the font size to
small and its shape to italic, and finally store the parameter value
in the macro definition \lstinline|\cl| (the parameter is only
accessible as \lstinline|#1| in the begin code).  Last comes the code
executed at the end of the environment: force a paragraph break,
produce just enough white space so that the citation is
right-justified, an em-dash and then the citation itself.

As a second example, consider the \lstinline|\msg| command, to typeset
one message in a protocol.  The protocol itself is best captured as a
list of messages.  To do this we define an environment, yet another
version of \lstinline|\msg| and a `and then do' command; these are
given in \autoref{lst:protocolex}.
\begin{lstlisting}[
  float,
  caption={An environment and a command to typeset protocols},
  label={lst:protocolex}
  ]
  \newcounter{msgnumber}
  \newenvironment*{protocol}
  { % begin code
    \setcounter{msgnumber}{0}%
    \newcommand*{\msg}[3]{%
      \refstepcounter{msgnumber}\themsgnumber&##1&##2&##3}
    \newcommand*{\next}{\\}
    \begin{math}\displaystyle%
      \begin{array}{r@{.\quad}l@{\rightarrow}l@{\;:\;}l}%
  }
  { % end code
      \end{array}%
    \end{math}%
  }
\end{lstlisting}
\newcommand{\enc}[2]{\{#2\}_{#1}}

Note that the definitions of the commands are made local to the
environment and cannot be accessed outside it (the counter declaration
must, alas, be global).  Because the definition of \lstinline|\msg| is
nested one level deep its parameters have names that start with
\emph{two} hashes, \lstinline|##1| and so on.

An example of their use is given in \autoref{lst:OtwayRees} (This
protocol is due to \citet{OtwayRees1987}; the notation $\enc{K}{M}$,
marked up as \lstinline|\enc{K}{M}|, means the encryption of $M$ under
symmetric key $K$).
\begin{lstlisting}[
  float,
  caption={Markup for the Otway-Rees protocol},
  label={lst:OtwayRees}
  ]
  \begin{protocol}
    \msg{A}{B}{M,A,B,\enc{K_{AS}}{N_{A},M,A,B}}
    \next
    \msg{B}{S}{M,A,B,\enc{K_{AS}}{N_{A},M,A,B},%
                     \enc{K_{BS}}{N_{B},M,A,B}}
    \next
    \msg{S}{B}{M,\enc{K_{AS}}{N_{A},K_{AB}},%
                 \enc{K_{BS}}{N_{B},K_{AB}}}
    \next
    \msg{B}{A}{M,\enc{K_{AS}}{N_{A},K_{AB}}}
  \end{protocol}
\end{lstlisting}
The typeset version is
\hyperref[thm:otwayrees]{Protocol~\ref*{thm:otwayrees}} on
Page~\pageref{thm:otwayrees}.

\cleardoublepage
\chapter{The body of the document}
\label{cha:body}


\section{The anatomy of the body}
\label{sec:bodyanatomy}

The body of the document has a structure given in
\autoref{lst:bodyanatomy}.
\begin{lstlisting}[
  float,
  caption={The anatomy of the body in \lstinline|UoYCSproject|},
  label={lst:bodyanatomy},
  morekeywords={listoffigures,listoftables,part,chapter,subsection,subsubsection,paragraph,subparagraph,appendix},
  basicstyle=\small
  ]
  % FRONT MATTER
  \listoffigures % Optional.  Generates a list of figures in the document.
  \listoftables  % Optional.  Generates a list of tables in the document.
  % Optional.  Other list-generating commands specific to your document.
  %            (For example, the listings package has a command
  %             \lstlistoflistings to produce a list of code listings.)
  % MAIN MATTER
  \part{title} % Repeat as often as necessary, perhaps zero times.
  \chapter{title} % Repeat as often as necessary, but at least once.
  \section{title} % Repeat as often as necessary, perhaps zero times.
  \subsection{title}% Repeat as often as necessary, perhaps zero times.
  \subsubsection{title}% Repeat as often as necessary, perhaps zero times.
  \paragraph{title}% Repeat as often as necessary, perhaps zero times.
  \subparagraph{title}% Repeat as often as necessary, perhaps zero times.
  % BACK MATTER
  \bibliography{file1,file2} % Construct bibliography from databases in
                             % `file1.bib' and `file2.bib'.
  \appendix % remaining chapters to be numbered as appendices
  \chapter{title} % Repeat as often as necessary, perhaps zero times.
  \section{title} % Repeat as often as necessary, perhaps zero times.
  \subsection{title} % Repeat as often as necessary, perhaps zero times.
  \subsubsection{title} % Repeat as often as necessary, perhaps zero times.
  \paragraph{title} % Repeat as often as necessary, perhaps zero times.
  \subparagraph{title} % Repeat as often as necessary, perhaps zero times.
\end{lstlisting}

There are usually three parts to a report:
\begin{description}
\item[Front matter] The title page, dedication, acknowledgements,
  abstract, tables of contents and so on.
  
  Most of this is taken care of automatically by the
  \lstinline|UoYCSproject| class (as long as you provide the
  declarations).  However, there are some optional features of the
  document (such as figures and tables) whose use cannot be detected.
  If you do use them you should indicate this by asking for the
  appropriate lists to be included.
\item[Main matter] The content of the document, appropriately
  structured.
  
  In \lstinline|UoYCSproject| the document is structured into
  chapters, with, optionally, a coarser structuring into parts (other
  classes have other rules).  The chapters can be structured into
  sections, the sections into subsections, and so on, using the
  commands given in \autoref{lst:bodyanatomy}.  You should not miss
  out a level of headings.  Note that \lstinline|\paragraph| and
  \lstinline|\subparagraph| are historical names that refer to titled
  sectional units, not to a coherent collection of sentences; a
  sectional (sub-)paragraph may well be composed of several coherent
  collections of sentences.
  
  Sections are numbered, and copied to the table of contents, as low
  as subsections. (If you really want to change the depth of the table
  of contents you can, although it is \emph{deprecated}, by altering
  the value of the \lstinline|tocdepth| counter.  For example,
  \lstinline|\setcounter{tocdepth}{3}| would cause sub-subsections to
  be numbered.  In the \lstinline|UoYCSproject| class you should do
  this in the local definitions file.)
  
  Sometimes a title will be too long for the table of contents or the
  running headings.  A shorter, optional, title can be given to the
  command; the short title is used instead of the long one in both the
  table of contents and the running headings.  For example, see
  \autoref{lst:shorttitle}.
  \begin{lstlisting}[
    float,
    caption={A sectional unit with an optional short title},
    label={lst:shorttitle},
    gobble={4}
    ]
    \chapter[The truth]{An accurate, complete and  verisimilitudinous %
      account of the happenings that occurred at that time and place}
  \end{lstlisting}
\item[Back matter] The references and appendices.

  Appendices are just chapters, although they will be numbered
  differently.
  
  There are various means of producing a bibliography or list of
  references.  The best way is through \BibTeX, a format for
  bibliographic databases that is integrated with \LaTeXe.
\end{description}

Not mentioned in \autoref{lst:bodyanatomy} are other parts of
documents usually found in the front or back matter, such as
glossaries and an index.  \LaTeXe\ has facilities to produce both of
these.  Only a glossary is worth including in a project report, and is
usually small enough to be done by hand as an appendix.  A good index
is very hard to produce, and not worth the trouble for a project
report (until you turn it into a book, that is!).

\section{Splitting the document up}
\label{sec:splitting}

Sometimes it is convenient to break a document into pieces.  \LaTeXe\
provides two mechanisms for doing this.

The command \lstinline|\input{<file>}| searches for a file called
`\textless file\textgreater.tex' and includes it.  The effect is as if
the file was typed in place.

The command \lstinline|\include{<file>}| searches for a file
called `\textless file\textgreater.tex' and includes it.  The file
should contain a complete chapter, and must start a new page.  The
\lstinline|\includeonly| command can be used to selectively process
chapters, speeding up processing time in the drafting phase.  Page
ranges and labels from the last run of missing chapters are taken
account of by this mechanism, so a small edit to one chapter may mean
only re-processing that chapter.  See the standard documentation.

\section{Text elements}
\label{sec:textelements}


\subsection{Modes}
\label{sec:modes}

\LaTeXe\ text is processed in various \emph{modes}.  The same input
will give different results in each mode.  The modes include:
\begin{description}
\item[paragraph] for ordinary text,
\item[left-to-right] for text that will not be broken across lines,
\item[math] for mathematics (actually there are two variants, in-line
  and displayed), and
\item[picture] for drawing simple pictures.
\end{description}
It is rare to be caught out by the wrong mode, as \LaTeXe\ usually
switches automatically when necessary, and most of the time you can
forget about modes.  The most common mistake is to use a command in a
text mode that only makes sense in math mode, when \LaTeXe\ will
report an error.  The reverse mistake ---to place text in a math
mode--- results in ugly output; see \autoref{fig:textasmath}.
\begin{figure}[tbp]
  \begin{center}
    \newcommand*{\phrase}{%
      Here is some text incorrectly placed in math mode --- note how %
      different it is from paragraph mode%
    }
    \fbox{\parbox{0.95\textwidth}{\phrase:\\\begin{math}\phrase\end{math}.}}
  \end{center}
  \caption{The result of treating text as mathematics}
  \label{fig:textasmath}
\end{figure}

\subsection{Simple paragraphs}
\label{sec:simpleparas}

A paragraph (in the sense of a coherent collection of sentences and
not in the sense of a sectional unit) is just a block of text.
Paragraphs are separated by blank lines (that is, sequences of at
least two new line characters).  Words in a paragraph are separated by
sequences of spaces and at most one newline.  See
\autoref{lst:quoteex}, where the first quotation consists of three
paragraphs.

Where necessary a paragraph break can be forced by a \lstinline|\par|
command.  The indentation on the first line of a paragraph can be
suppressed by beginning the paragraph with a \lstinline|\noindent|
command.

\subsection{Characters}
\label{sec:characters}

\subsubsection{Reserved characters}
\label{sec:reserved}

There are some characters which are reserved and may not be used in
text.  These are listed in \autoref{tab:forbidden}, together with how
to make them if you really need them.
\begin{table}[tbp]
  \centering
  {\small
    \begin{tabular}{*{10}{|c}|}
      \hline
      \#&\$&\%&\&&\textasciitilde&\_&\^{}&\textbackslash&$\{$&$\}$
      \\\hline
      \lstinline|\#|&\lstinline|\$|&\lstinline|\%|
      &\lstinline|\&|&\lstinline|\textasciitilde|
      &\lstinline|\_|&\lstinline|\^{}|
      &\lstinline|\textbackslash|&\lstinline|\{|&\lstinline|\}|
      \\\hline
    \end{tabular}
    }
  \caption[Reserved characters and how to make them]{Reserved
    characters and how to make them.  Note that the two braces are
    only defined in math mode.}
  \label{tab:forbidden}
  \rule{\textwidth}{1pt}
\end{table}


\subsubsection{Ellipses}
\label{sec:ellipses}

Sometimes you will need to show that words have been left out of a
quotation.  This is done by a mark called an \emph{ellipsis} `\ldots';
it can be made by the `low dots' command \lstinline|\ldots|.  The
output of \lstinline|\ldots| is not the same as three full stops:
compare a\ldots to...z.  It is bad style to let a sentence trail off
with an ellipsis\ldots

For mathematics, centred dots (\lstinline|\cdots|) look better:
$1+\frac{1}{2}+\cdots+\frac{1}{2^n}+\cdots+\frac{1}{256}$.  (Some
people think that $\sum_{n=0}^{8}\frac{1}{2^{n}}$ looks even better.)

A vertical ellipsis can be made with \lstinline|\vdots|; an example of
its use can be seen in \autoref{sec:bibliographies}.

\subsubsection{Dashes}
\label{sec:dashes}

Another class of characters that sometimes causes confusion are the
various dashes.  See \autoref{tab:dash}.
\begin{table}[tbp]
  \centering
  \newcommand{\hding}[1]{\multicolumn{1}{c}{\textbf{#1}}}
  \begin{tabular}{lcllp{0.3\textwidth}}
    \hding{Name}&\hding{Character}&\hding{Mark-up}&\hding{Mode}&\hding{Comment}
    \\
    Hyphen&-&\lstinline|-|&Text&To join two words, as in `Kraft-Ebbing'.
    \\
    en-dash&--&\lstinline|--|&Text&To form a range, as in `The
    period 1997--2003'.
    \\
    em-dash&---&\lstinline|---|&Text&To separate two phrases --- or use as
    parenthesis brackets.
    \\
    Minus sign&$-$&\lstinline|-|&Math&To indicate subtraction, as in
    `$2003-1997$'.
  \end{tabular}
  \caption{Dashes and their use}
  \label{tab:dash}
  \rule{\textwidth}{1pt}
\end{table}

Many, many special characters and symbols are available.  Some are
available automatically, some are parts of packages that you will need
to load explicitly.  See `The Comprehensive \LaTeX\ Symbol List'.

\subsubsection{Spaces}
\label{sec:spaces}

\lstset{showspaces={true}}



Spaces are a special case (in this section space characters are
typeset thus: `\lstinline| |').  There are three kinds:
\begin{enumerate}
\item `\lstinline| |'\quad An ordinary space (or a non-empty sequence
  of spaces). May be printed as an inter-word space, an inter-sentence
  space or a newline.  Under certain circumstances (for example, in a
  math mode or immediately following a command) it may be ignored.

  Most of the time you should use ordinary spaces to separate items.
\item `\lstinline|~|'\quad A \emph{tie}.  It will
  never be replaced by anything other than an inter-word space.
  
  Ties should be used whenever you want to suppress a line-break.  In
  particular, they should be used in constructs such as
  \lstinline|Mr~Smith|, \lstinline|Hypothesis~C|, and so on.
\item `\lstinline|\ |' A hard space.  It may be replaced by an
  inter-word space or a newline.  It may not be ignored.

  Hard spaces are useful when you want to force a new line, and \TeX\
  does not think it is building a line: use \lstinline|\ \newline|.
  
  A hard space is used to protect an inter-word space immediately
  following a command name.  For example `\lstinline|\TeX is useful|'
  typesets as `\TeX is useful', while `\lstinline|\TeX\ is useful|'
  typesets as `\TeX\ is useful'.  (An alternative method is to place
  an empty pair of braces after the command, for example
  `\lstinline|\TeX{} is useful|'; this has the advantage that it works
  no matter what the following character, for example
  `\lstinline|\TeX{}: useful!|'.)
  
  Their other use is to prevent \LaTeXe\ from thinking it is at a
  sentence end when it is not.  \LaTeXe\ (because \TeX\ does) treats a
  space as an inter-sentence space if it is preceded by a
  non-uppercase character and a full-stop.  This can happen with
  abbreviations, e.\ g.\ `etc.'.  (Compare the spaces after the `e.'
  and the `g.' with the inter-word and inter-sentence space on the
  same line.)
\end{enumerate}

\lstset{showspaces={false}}

\subsubsection{Character attributes}
\label{sec:cahrattributes}

It is possible to vary the series, shape, family, size and colour of
\emph{text} fonts (see the references for the attributes for
mathematical fonts).  This is \emph{deprecated} in the text, but
recommended in the implementation of abstract syntax.  See
\autoref{tab:fontstyle}.
\begin{table}[tbp]
  \centering
  \begin{tabular}{lll}
      Series&\lstinline|\mdseries|&\normalfont\mdseries Medium Series
    \\      &\lstinline|\bfseries|&\normalfont\bfseries Boldface Series
    \\Family&\lstinline|\rmfamily|&\normalfont\rmfamily Roman Family
    \\      &\lstinline|\sffamily|&\normalfont\sffamily San Serif Family
    \\      &\lstinline|\ttfamily|&\normalfont\ttfamily Typewriter Family
    \\Shape &\lstinline|\upshape| &\normalfont\upshape  Upright Shape
    \\      &\lstinline|\itshape| &\normalfont\itshape  Italic Shape
    \\      &\lstinline|\slshape| &\normalfont\slshape  Slanted Shape
    \\      &\lstinline|\scshape| &\normalfont\scshape  Small Caps Shape
  \end{tabular}
  \caption{Font attribute declarations}
  \label{tab:fontstyle}
\end{table}
There is also a \lstinline|\normalfont| declaration when all else
fails.

To each font declaration there is a command, of the form
\lstinline|\textXX{text}|, where the \lstinline|XX| should be replaced
by the first two letters of the corresponding declaration: for
example, `\lstinline|\textsc{text}|' produces `\textsc{Text}'.  The
exception to the rule is `\lstinline|\textnormal{text}|'.

Font size is controlled by the declarations given in
\autoref{tab:fontsize}.  (Not all font sizes may be available;
if not available something close will be chosen.)
\begin{table}[tbp]
  \newcommand{\fs}[1]{\csname#1\endcsname abcXYZ}
  \centering
  \begin{tabular}{llll}
    \lstinline|\tiny|&\lstinline|\scriptsize|&\lstinline|\footnotesize|&\lstinline|\small|
    \\
    \fs{tiny}&\fs{scriptsize}&\fs{footnotesize}&\fs{small}
    \\[1ex]
    \lstinline|\normalsize|&\lstinline|\large|&\lstinline|\Large|&\lstinline|\LARGE|
    \\
    \fs{normalsize}&\fs{large}&\fs{Large}&\fs{LARGE}
    \\[1ex]
    \lstinline|\huge|&&\lstinline|\Huge|&
    \\
    \multicolumn{2}{l}{\fs{huge}}&\multicolumn{2}{l}{\fs{Huge}}
  \end{tabular}
  \caption{Font size declarations}
  \label{tab:fontsize}
  \rule{\textwidth}{1pt}
\end{table}

To change colours you need to load the \lstinline|color| package.  You
get a declaration, \lstinline|\color{colour}| and its associated
command \lstinline|\textcolor{colour}{text}|.  You also get coloured
backgrounds and framed boxes.  A few colours
({\newcommand{\tc}[1]{\textcolor{#1}{#1}}\tc{red}, \tc{blue},
  \tc{green}, \tc{cyan}, \colorbox{black}{\tc{yellow}}, \tc{magenta},
  \tc{black}, \colorbox{black}{\tc{white}}}) are pre-defined; you must
define others yourself.  \emph{The use of colour is deprecated: if you
  must use it, do so very, very carefully.}

\subsection{Emphasised text}
\label{sec:emphasis}

Emphasised text should be marked up logically, using
\lstinline|\emph|.  The command is context dependent and can be
nested.  For example,\footnote{The first sentence of
  \href{http://www.cs.utexas.edu/users/EWD/ewd04xx/EWD498.PDF}{EWD498},
  but with my emphasis!}
\begin{quote}
  \lstinline[basicstyle=\sffamily\small]|\textnormal{Sometimes \emph{we \emph{discover} unpleasant} truths.}|
\end{quote}
typesets as
\begin{quote}\normalfont
  Sometimes \emph{we \emph{discover} unpleasant} truths.
\end{quote}

Use of \lstinline|\textit{...}| or \lstinline|\itshape| to simulate
the same effect is deprecated.

\subsection{Lists}
\label{sec:lists}

\LaTeXe\ has three kinds of lists available:
\begin{description}
\item[bulleted] made with the \lstinline|itemize| environment,
\item[numbered] made with the \lstinline|enumerate| environment, and
\item[labelled] made with the \lstinline|description| environment
  (this list is an example of the \lstinline|description|
  environment).
\end{description}
Each has a similar format.  Individual items are introduced by
\lstinline|\item|; in the case of the \lstinline|description|
environment the \lstinline|\item| command has an `optional' parameter
for the label (which \emph{must} be present).  \LaTeXe\ changes the
numbering and bulleting styles for sub-lists, to a reasonable depth
(if you exceed this it probably means you have a poorly structured
document).  See \autoref{lst:lists}.
\begin{lstlisting}[
  float,
  caption={Examples of lists},
  label={lst:lists}
  ]
  \begin{description}
  \item[Thing One] likes
    \begin{enumerate}
    \item Green Eggs and
    \item Ham
    \end{enumerate}
  \item[Thing Two] has never seen either
    \begin{itemize}
    \item a Star-Bellied Sneetch or
    \item a Lorax.
    \end{itemize}
  \end{description}
\end{lstlisting}

It is possible to define your own list structures, and this is a
common way of building an abstract syntax for a document-specific
structure (for example, a variant of the enumerated list would have
been a good way to build a protocol display).  See a standard
reference (\autoref{sec:books}).


\subsection{Quotations}
\label{sec:quotations}


\subsubsection{In-line quotations}

A running quotation in text must be surrounded by quote marks.  There
are two kinds, double and single.  Opening single quotes are made with
the ``\texttt{`}'' character, and the corresponding close quote is
made with the ``\texttt{'}'' character.  Double quotes are made with
\emph{pairs} of single quotes: ``\texttt{``}'' and ``\texttt{''}'';
the double-quote character ``\verb|"|'' is never
used.\footnote{Actually, it is used.  One use is in the sentence to
  which this footnote is attached.  Another use is in code listings
  for programming languages that, for example, use the character to
  delimit strings.  It is also the name of the command that produces
  the `\"{ }' accent in words such as `co\"{o}rdinate'; accents are
  not discussed in this document.}

\subsubsection{Displayed quotations}

There are two kinds of displayed quotation:
\begin{enumerate}
\item the \lstinline|quote| environment, and
\item the \lstinline|quotation| environment.
\end{enumerate}
The two environments are very similar.  The \lstinline|quote|
environment is recommended for short quotes and the
\lstinline|quotation| environment for long quotes.  Both are
illustrated in \autoref{lst:quoteex}.

\subsubsection{WARNINGS}
\newcommand{\beware}{\marginpar[\textcolor{red}{\textsc{Beware}
    $\Rightarrow$}]{\textcolor{red}{$\Leftarrow$ \textsc{Beware}}}}
\begin{itemize}
\item Neither of the displayed quotation environments adds quotation
  marks.\beware
  
  Check the Student Handbook to find out if we currently require
  quotation marks around displayed quotes.  If they are required you
  will need to add them manually.
\item Departmental rules require a citation with all quotes.  \beware
  These must be supplied manually.  See \autoref{lst:quoteex} and
  \autoref{lst:mqex} for examples.
\end{itemize}

\subsection{Bibliographies}
\label{sec:bibliographies}

Various packages have been written to enhance the presentation of
bibliographies and citations.  The \lstinline{UoYCSproject} class
loads the \lstinline{natbib} style and sets up citations to follow the
Departmental approved style (IEEE).  The \lstinline{UoYCSproject}
class also fixes the bibliography style for the approved departmental
style.

Lists of references can be generated from a database in \BibTeX\
format.  This is a flat text file.  The documentation for the IEEEtran
\BibTeX{} styles \cite{Shell2008} will tell you how to format this
file.  The references for this document are an example.  Each entry
has the following layout:
\begin{quote}
  \sffamily
  \begin{tabular}{l}
    @Entry\_Type$\{$Label,\\
    \begin{tabular}{rcl}
      Field\textsubscript{0} &=& $\{$Value\textsubscript{0}$\}$,\\
      Field\textsubscript{1} &=& $\{$Value\textsubscript{1}$\}$,\\
      &\vdots&\\
      Field\textsubscript{n} &=& $\{$Value\textsubscript{n}$\}$
    \end{tabular}\\
    $\}$
  \end{tabular}
\end{quote}
There are many different entry types and each type has a different
array of compulsory and optional fields.  There are two features of
\BibTeX{} that cause problems when preparing the bibliographic
database: see \autoref{sec:BibTeXgotchas}.

Citations are of two types, parenthesized and textual.  If
\lstinline|Joyce:FW| is a label associated with the record for James
Joyce's \emph{Finnegans Wake} then
\begin{description}
\item[Parenthesized] 
  \begin{tabular}[t]{l@{~\textrm{generates}~}l}
    \lstinline|\citep{Joyce:FW}|&\citep{Joyce:FW}
    \\
    \lstinline|\citep[\S4]{Joyce:FW}|&\citep[\S4]{Joyce:FW}
    \\
    \lstinline|\citep[see][\S4]{Joyce:FW}|&\citep[see][\S4]{Joyce:FW}
  \end{tabular}
\item[Textual]
  \begin{tabular}[t]{l@{~\textrm{generates}~}l}
    \lstinline|\citet{Joyce:FW}|&\citet{Joyce:FW}
    \\
    \lstinline|\citet[\S4]{Joyce:FW}|&\citet[\S4]{Joyce:FW}
    \\
    \lstinline|\citet[see][\S4]{Joyce:FW}|&\citet[see][\S4]{Joyce:FW}
  \end{tabular}
\end{description}

If several citations are applicable they can be included in the same
citation command, as a comma separated list, without spaces; for
example:
`\ldots\lstinline|important modernist works~\citep{Elliot:WL,Joyce:FW}|'
might produce\\`\ldots important modernist works~[17, 18]'.

The \lstinline|natbib| package provides several other facilities for
typesetting parts of citations, such as titles; see its documentation.

\subsection{Floats}
\label{sec:floats}

A \emph{float} is a numbered item, usually with a caption, that can
`float' around the document and gets a special entry in the front
matter.  In this document there are three classes of float,
\emph{tables} (for example, \autoref{tab:UoYCSpdeclarations}),
\emph{figures} (for example, \autoref{fig:textasmath}) and
\emph{listings} (for example, \autoref{lst:quoteex}). The contents of
the item need have no relation to the class of float, although it is
helpful to the reader if they do!

Control of floats and their positioning is a complex subject, and
apart from mimicking examples in the source of this document you
really ought to consult a standard reference (see
\autoref{sec:books}).

Each float can have a symbolic label for use by \LaTeXe's
cross-referencing mechanism.


\subsection{Tabulating data}
\label{sec:tabular}

The common thing to find in a table float is a \lstinline|tabular|
environment.  These are very flexible, and too complex to describe in
this note.  Simple examples may be seen in \autoref{tab:forbidden} and
\autoref{tab:UoYCSpdeclarations}.  There are also packages to give
tabular environments with extra functionality.  See a standard
reference (\autoref{sec:books}).

\subsection{Theorem-like environments}
\label{sec:theorems}

\LaTeXe\ allows you to define special series of named and numbered
paragraphs, called theorem-like environments.  The canonical examples
are theorems, lemmata and hypotheses.  Theorem-like environments have
an optional parameter for textually naming the content of the
environment, in addition to numbering it.

In \autoref{lst:newtheoremex} I define one for numbering protocols.
My usual habit is to combine a theorem-like environment with an
ordinary one to get a `close-paragraph' marker; and I have done that
here.  (A better solution would be to roll the \lstinline|protocol|
and \lstinline|prot| environments together; I have separated them here
for illustration.)
\begin{lstlisting}[
  float,
  caption={A theorem-like environment for protocols},
  label={lst:newtheoremex},
  morekeywords={ifthenelse}
  ]
  \newtheorem{PROTOCOL}{Protocol}
  \newenvironment{prot}[1][]
  {\newcommand*{\tmp}{#1}
    \ifthenelse{\equal{\tmp}{\empty}}
       {\begin{PROTOCOL}}
       {\begin{PROTOCOL}[\tmp]\ \newline}
  }
  {\newline\hspace*{\fill}
    \rule{0.666666em}{1.07867788em} % Golden ratio (approx)
  \end{PROTOCOL}}
\end{lstlisting}
As an example here is \autoref{lst:OtwayRees}, typeset using
\begin{quote}
\lstinline|\begin{prot}[Otway-Rees]|\ldots\lstinline|\end{prot}|.
\end{quote}

\newtheorem{PROTOCOL}{Protocol}
\newenvironment{prot}[1][]
{\newcommand*{\tmp}{#1}
  \ifthenelse{\equal{\tmp}{\empty}}
    {\begin{PROTOCOL}}
    {\begin{PROTOCOL}[\tmp]\ \newline}
}
{   \newline\hspace*{\fill}
    \rule{0.666666em}{1.07867788em} % Golden ratio (approx)
  \end{PROTOCOL}
}
\newcounter{msgnumber}
\newenvironment*{protocol}
{
  \setcounter{msgnumber}{0}%
  \newcommand*{\msg}[3]{\refstepcounter{msgnumber}\themsgnumber&##1&##2&##3}
  \newcommand*{\next}{\\}
  \begin{math}\displaystyle%
    \begin{array}{r@{.\quad}l@{\rightarrow}l@{\;:\;}l}%
}
{
    \end{array}%
  \end{math}%
}

\begin{prot}[Otway-Rees]\label{thm:otwayrees}
    \begin{protocol}
    \msg{A}{B}{M,A,B,\enc{K_{AS}}{N_{A},M,A,B}}
    \next
    \msg{B}{S}{M,A,B,\enc{K_{AS}}{N_{A},M,A,B},\enc{K_{BS}}{N_{B},M,A,B}}
    \next
    \msg{S}{B}{M,\enc{K_{AS}}{N_{A},K_{AB}},\enc{K_{BS}}{N_{B},K_{AB}}}
    \next
    \msg{B}{A}{M,\enc{K_{AS}}{N_{A},K_{AB}}}
  \end{protocol}
\end{prot}


\subsection{Mathematics}
\label{sec:mathematics}

\LaTeXe\ has superb type-setting facilities for mathematics (see
\autoref{fig:mathematics}).  For complex work the
$\mathcal{A_{\displaystyle \!M\!}S}$\TeX{} packages (developed by the
American Mathematical Society) will handle everything you could
possibly need.  Most people only need the basic, pre-loaded \LaTeXe{}
facilities.\footnote{There is a school of thought that mistakes were
  made in the basic, pre-loaded \LaTeXe{} facilities, particularly the
  \lstinline{eqnarray} environment.  The solution proposed by this
  school is to always load and use the $\mathcal{A_{\displaystyle
      \!M\!}S}$\TeX{} packages.}
\begin{figure}[tbp]
  \begin{displaymath}
    \int^{1}_{-1}\!\frac{(T_{n}(x))^{2}}{\sqrt{1-x^{2}}}\,dx=\left\{
        \begin{array}{l@{\;\;\mbox{if}\;\;}l}
          \pi&n=0
          \\\pi/2&n\in\mathbb{N}_{1}
        \end{array}
      \right.
  \end{displaymath}
  \caption[An example of mathematical type-setting]{An example of
    mathematical type-setting.  (Othogonality in Chebyshev polynomials;
    $T_{n}$ is the nth Chebyshev polynomial:
    $T_{n}(x)=\cos(n\cos^{-1}x)$.)}
  \label{fig:mathematics}
  \rule{\textwidth}{1pt}
\end{figure}

In-line mathematics can be produced using the `math-shift'
construction, \lstinline{$}\ldots\lstinline{$}.  For example,
\lstinline|$A+B$| produces `$A+B$'.  An alternative is to use the
\lstinline{math} environment.

Displayed mathematics is made using the \lstinline{displaymath},
\lstinline{equation}, \lstinline{eqnarray} and \lstinline{eqnarray*}
environments, depending on the exact effect desired.

The subject is too complex to discuss here, and the standard
references (\autoref{sec:books}) should be consulted.


\subsection{Cross references}
\label{sec:crossrefs}

\LaTeXe\ has a powerful cross-reference mechanism.  Anything for which
a number can be generated (parts, chapters, sections, tables, figures,
theorem-like structures, equations and so on) can have a symbolic
label.  See \autoref{lst:label}, where a section is given the label
\lstinline|sec:brief|.
\begin{lstlisting}[
  float,
  caption={An example of a label},
  label={lst:label}
  ]
  \section[Brief titles]{How to avoid tedious prolixity in the titles
    of sections, when they are printed in the Table of Contents}
  \label{sec:brief}
\end{lstlisting}

The section number can be referred to by using
`\lstinline|\ref{sec:brief}|'.  You can also refer to the page on
which the section occurs by using the command
`\lstinline|\pageref{sec:brief}|'. (See the examples in
\autoref{lst:ref}.)
\begin{lstlisting}[
  float,
  caption={Examples of cross-references},
  label={lst:ref}
  ]
  In Section~\ref{sec:brief}, starting on Page~\pageref{sec:brief}, we
  see how to do it.

  Sections~\ref{sec:long}--\ref{sec:brief} report this in detail.
\end{lstlisting}



The \lstinline|hyperref| package (loaded as part of
\lstinline|UoYCSproject|) automatically turns references made with
\lstinline|\ref| and \lstinline|\pageref| into internal links when a
suitable format is output (for example, PDF).  It also adds three
further commands:
\begin{itemize}
\item \lstinline{\ref*}, which does \emph{not} make the internal
  hyperlink.
\item \lstinline{\autoref}, which (sometimes) adds the name of the
  type of unit; for example the command
  \lstinline|\autoref{sec:brief}| will generate `section~3.2' (or what
  ever number it turns out to be).  The \lstinline|hyperref| package
  makes the whole phrase into an internal link.
  
  (If \lstinline|\autoref| causes problems the easiest thing to do is
  fall back on \lstinline|\ref| and await the bug fixes.)
\item \lstinline{\nameref}, which typesets the name of the section.
\end{itemize}


\subsection{Pictures}
\label{sec:pictures}

Pictures in \LaTeXe\ can either be
\begin{itemize}
\item drawn within \LaTeXe's native \lstinline|picture| environment
  (if they are simple)
\item drawn by a more sophisticated package, such as
  \lstinline|pgf/tikz| and \lstinline|pdftricks|, or
\item imported using an external format, using a package such as
  \lstinline|graphics| (you need to be careful with formats;
  \lstinline|pdflatex| cannot accept PostScript, Encapsulated or
  otherwise; PNG or PDF works best\footnote{There is a Linux program
    to convert from Encapsulated PostScript to PDF,
    \textsf{epstopdf}.}).
\end{itemize}
See the documentation given in \autoref{cha:usefulrefs}.

\cleardoublepage
\part{The document class UoYCSproject}
\label{sec:end}
\thispagestyle{empty}\cleardoublepage

\chapter{The document class UoYCSproject}
\label{cha:UoYCSproject}

\section{The antecedents of UoYCSproject}
\label{sec:uoycsp:ante}

The \LaTeXe\ class \lstinline|UoYCSproject| is based on the
\KOMAScript\ class \lstinline|scrreprt| and so has most of the
facilities provided by that class.  However some, such as page layout
and the title declarations, are fixed or redefined.  (For the record,
the following options are passed to \lstinline|scrreprt|:
\lstinline|12pt|, \lstinline|a4paper|,
\lstinline|twoside|\footnote{You must print a document of class
  \lstinline|UoYCSProjct| double-sided.}, \lstinline|abstracton|,
\lstinline|pointlessnumbers|, \lstinline|BCOR13mm|.)

The \lstinline|ifthen| package is provided.

\lstinline|UoYCSproject| chooses the font encoding (\lstinline|T1|)
using the \lstinline|fontenc| package and font sets by packages from
the PSNFSS bundle \citep{Schmidt2004} (for roman shape: Hermann Zapf's
Palatino, the University's font, using the \lstinline|mathpazo|
package; for san serif shape: Helvetica, using the \lstinline|helvet|
package with a scaling of \lstinline|0.9|; for typewriter shape:
Courier, using the \lstinline|courier| package), while accessing the
micro-typographic features of \lstinline|pdfetex| (character protusion
and font expansion) via the \lstinline|microtype| package.

British English hyphenation and names are set, using the
\lstinline|babel| package.  (If you need them, \lstinline|babel|
allows you to include other languages in your document.)

The bibliography and citation styles are fixed using
\lstinline{natbib}, setting the bibliography style to
\lstinline{IEEEtranN}.

Hyperlinks are produced using the \lstinline|hyperref| package.  This
package also produces bookmarks and sets some of the PDF `document
properties'.  Anchor placement in floats is improved by loading the
\lstinline{hypcap} package, with parameter \lstinline{all}.

The \lstinline{UoYCSproject} class works with the versions available
as part of the \href{http://www.tug.org/texlive/}{\TeX-Live~2007
  distribution} (\url{http://www.tug.org/texlive/}).

\section{Declarations for the title pages}
\label{sec:uoycsp:decs}

The available declarations are listed in
\autoref{tab:UoYCSpdeclarations}.
\begin{table}[tbp]
  \centering
  \begin{tabular}{lcc}
    \textbf{Declaration}&\textbf{Parameter}&\textbf{Optionality}\\
    \lstinline|\title|&\lstinline|{short text}|&C\\
    \lstinline|\author|&\lstinline|{short text}|&C\\
    \lstinline|\date|&\lstinline|{short text}|&O\\
    \lstinline|\abstract|&\lstinline|{long text}|&C\\
    \lstinline|\wordcount|&\lstinline|{short text}|&C\\
    \lstinline|\includes|&\lstinline|{short text}|&O\\
    \lstinline|\excludes|&\lstinline|{short text}|&O\\
    \lstinline|\dedication|&\lstinline|{short text}|&O\\
    \lstinline|\acknowledgements|&\lstinline|{long text}|&O\\
    \lstinline|\BEng|&---&1\\
    \lstinline|\BSc|&---&1\\
    \lstinline|\MEng|&---&1\\
    \lstinline|\MMath|&---&1\\
    \lstinline|\SWE|&---&1\\
    \lstinline|\SCSE|&---&1\\
    \lstinline|\MIT|&---&1\\
    \lstinline|\MNC|&---&1\\
    \lstinline|\GTC|&---&1
  \end{tabular}
  \caption[Declarations of class UoYCSproject]{%
    Declarations of class UoYCSproject.

    The declarations typeset in \textsf{\textbf{bold, san-serif font}}
    are common to many classes; the remainder are peculiar to
    UoYCSproject.  Where declarations take parameters the
    type of the parameter, short (paragraph breaks forbidden) or long
    (paragraph breaks allowed) is given.

    The optionality tags have the following meanings:  `C':
    compulsory; `O': optional; `1': choose exactly one of this group.
  }
  \label{tab:UoYCSpdeclarations}
\end{table}

The \lstinline|\title|, \lstinline|\author| and \lstinline|\date|
declarations are standard.  You should use them to record: the
\emph{title of your report}, \emph{your name} and the \emph{date of
  submission} respectively.  If the date is omitted a message giving
the date of processing is produced; this should not be on your final
submission!

You are required to produce an abstract.  Most classes achieve this by
an \lstinline|abstract| environment in the body (including the
\KOMAScript\ classes).  This is changed by \lstinline|UoYCSproject| to
an \lstinline|\abstract| declaration in the preamble.

You also need to give the word count of the parts of the document to
be marked.  There is a compulsory declaration, \lstinline|\wordcount|,
to state the actual word count of the main body of the
report.\footnote{Under Un*x you can do this by running
  %
  \lstinline/wc -w/ on the file.
  %
  If you split the document between files you can use a command of the
  pattern
  %
  \lstinline/cat file1 file2 file3 | wc -w/.
  %
  An alternative is to use the \TeX{}count utility (see
  \url{http://tug.ctan.org/pkg/texcount}) which has a web interface at
  \url{http://folk.uio.no/einarro/Services/texcount.html}.} Optionally
you can generate text that states which extra sections are included,
and which excluded by the \lstinline|\includes| and
\lstinline|\excludes| declarations.  If both optional declarations are
omitted the message produced is:
\begin{quote}
  ``This includes the body of the report only.''
\end{quote}
If the inclusions only are given, the message produced is:
\begin{quote}
  ``This includes the body of the report, and \textless include
  text\textgreater.''
\end{quote}
If the exclusions only are given, the message produced is:
\begin{quote}
  ``This includes the body of the report, but not \textless exclude
  text\textgreater.''
\end{quote}
If both are given the message produced is:
\begin{quote}
  ``This includes the body of the report, and \textless include
  text\textgreater, but not \textless exclude text\textgreater.''
\end{quote}

You should also state which qualification the project contributes to
by using exactly one of the declarations: \lstinline|\BEng|,
\lstinline|\BSc|, \lstinline|\MEng|, \lstinline|\MMath|,
\lstinline|\SWE|, \lstinline|\MIT|, \lstinline|\GTC| or
\lstinline|\SCSE| (\lstinline|\MIP| is available for historical
purposes|).  These take no parameter.

You may generate a page with a dedication and/or acknowledgements on
it by using the declarations \lstinline|\dedication| and
\lstinline|\acknowledgements|.

Users of the \lstinline|\include| mechanism may add an
\lstinline|includeonly| declaration.

The title pages are typeset in the usual way, by a
\lstinline[morekeywords={maketitle}]{\maketitle} command as the first
command in the body of the document.

\section{Loading your own packages and adding your own commands}
\label{sec:uoycsp:diy}

Because \lstinline|UoYCSproject| needs to carefully control the order
of package loading you should include nothing in the preamble other
than the declarations given in \autoref{sec:uoycsp:decs}.

A non-standard mechanism is provided for loading your own packages and
declaring your own commands and environments.  If your main file is
called \lstinline|<main>.tex|, then the extra preamble should go in a
file called \lstinline|<main>.ldf| (for \emph{L}ocal
\emph{D}e\emph{f}initions).

\section{Other non-standard facilities}
\label{sec:nonstandard} 

\subsection{Citations}
\label{sec:citations}

The citation mechanism in \lstinline{UoYCSproject} is different from
the standard, which uses the command \lstinline{\cite}.  It uses the
more flexible scheme implemented by the \lstinline{natbib} package, of
\lstinline{\citep} for parenthesised citations and \lstinline{\citet}
for citations as text.  See \autoref{sec:bibliographies}.

The command \lstinline{\cite} is defined to be the same as
\lstinline{\citet} (which is probably not what you want).

\subsection{Cross references}
\label{sec:labels}

The standard mechanism (\lstinline|\ref{label}|) works, but
\lstinline|\autoref{label}| is preferred.  The \lstinline{\autoref}
command generates the location type (section, subsection, or whatever)
as well as the location number.  See \autoref{sec:crossrefs}.

\bibliography{references}

\appendix
\cleardoublepage
\part{Appendices}
\thispagestyle{empty}\cleardoublepage

\chapter{Packages not pre-loaded that you may find useful}
\label{cha:usefulpackages}

These are packages that might help you with special tasks in writing
your report.  (I have omitted specialist packages that some people
might find useful, such as the package which provides support for
Braille.)

If these packages are not in our standard \TeX-Live release they can
be obtained from the Common \TeX{} Archive Network (CTAN); this is
easiest via the catalogue
(\url{http://www.tex.ac.uk/tex-archive/help/Catalogue/}).  Even if we
do have the packages, you may wish to check for later versions.

\section{Main document}

\begin{description}
\item[array] Improves the facilities for tabular and array
  environments. 
\item[acronym] Helps you manage acronyms, ensuring that
  all are printed in full at least once.  It can generate a list of
  used acronyms, too.
\item[amsmath] Enhanced mathematical type-setting; there are several
  ancillary packages.
\item[calc] Allows easier arithmetic calculations than native mode.
  The documentation comes with a syntax, formal semantics and
  implementation scheme, as well as an informal narrative.
\item[changebar] Allows you to indicate changes to your document by a
  bar in the margin.  (Useful for showing drafts to your supervisor.)
\item[glossaries] To aid production of a glossary.
\item[graphics] To include pictures from external sources
\item[graphicx] Like `graphics', but with a `key=value' interface.
\item[listings] To pretty-print code listings.  Several languages are
  predefined, and you can define your own.
\item[movie15] To insert moving images in the PDF.  Particularly
  useful for presentations of physical artefacts (see
  \autoref{sec:presentationpackages}).
\item[pdfcomment] Allows you to take advantage of the PDF comment and
  annotation facilities, for on-line copies but \emph{not} the printed
  copy.  (PDF comments are not widely supported outside of Adobe
  products.)
\item[pdfpages] Allows you to include a PDF document inside your
  \LaTeXe\ document.  It is very flexible, allowing you to select
  pages, print \emph{n} logical pages per physical page, and so on.
\item[pgf/tikz] Allows more complex drawings than native \LaTeXe{}
  mode.  Suitable for all flavours of \LaTeXe{}.  There is a very well
  designed font end to \lstinline{pgf}, called \lstinline{TikZ}, that
  makes drawing diagrams much easier.
\item[siunitx] For consistent typesetting of physical quantities.
\item[todonotes] Allows you to insert `to do' markers that are
  visually obvious, and to generate a table-of-contents-like list of
  them.
\end{description}

\section{Presentations}
\label{sec:presentationpackages}

You may need to give a presentation, and there are several \LaTeXe{}
packages for preparing slides; a good list may be found at
\url{http://www.miwie.org/presentations/}.

The main advantage of using a \LaTeXe{}-based solution for
presentations is being able to re-use your \LaTeXe{} source and avoid
re-typing everything for \lstinline|PowerPoint| (or similar
presentation tool, such as the one in \lstinline|OpenOffice|).  The
package of choice for most people is \lstinline|beamer|.  Section~5 of
the Beamer \emph{User's Guide} gives a lot of good advice on creating
presentations, even (or especially) if you opt to use
\lstinline|PowerPoint|.

\cleardoublepage
\chapter{Common \LaTeXe\ `Gotchas'}
\label{cha:gotchas}

There are just a few things that trip up a newcomer to \TeX{},
\LaTeXe{} and \BibTeX.

\section{Parameterless macros gobble white space}

\subsection{Problem}
Any macro gobbles all the white space following up to the next
non-white space character.  This does not matter if the next thing is
a parameter, but it does matter otherwise.  For example,
`\lstinline[showspaces]{\LaTeXe is easy.}' typesets as `\LaTeXe is
easy.'.

\subsection{Solution}
A solution is to always protect the white space by preceding it with a
backslash: `\lstinline[showspaces]{\LaTeXe\ is easy.}'.  Another
solution is to always follow a parameterless macro with empty braces:
`\lstinline[showspaces]/\LaTeXe{} is easy./'.  Both typeset as
`\LaTeXe{} is easy.'.  The second solution is more robust if the white
space is replaced by something else, such as punctuation.

\section{Confusion between end of abbreviation and end-of sentence}

\subsection{Problem}

\TeX{} treats a full stop, `.' between a non-capital letter and white
space as indicating an end of a sentence, and so it generates a
sentence-separating space rather than a word-separating space.  The
problem most commonly arises with abbreviations such as `etc.', `i.\
e.', `e.\ g.' and so on.\footnote{Some people think it is better style
  to use `and so on', `that is' and `for example', and so on, neatly
  avoiding the problem; it also avoids the quite common confusion
  between `i.\ e.' and `e.\ g.'.}

\subsection{Solution}

Protect any such spaces with a backslash.  \lstinline[showspaces]{\ }
is always treated as an inter-word space.  See \autoref{sec:spaces}.

\section{Wrong type of dash}

There are four different types of dash available in \TeX-based
systems. The different dashes have different uses: see
\autoref{sec:dashes} for a discussion.

\section{Wrong type of quote}

In \TeX-based systems quotation marks come in balanced pairs:
\begin{itemize}
\item `6-9 quote marks' (enlarged: {\Huge`}\ldots{\Huge'})
\item ``66-99 quote marks'' (enlarged: {\Huge``}\ldots{\Huge''})
\end{itemize}
None of these are made using the `\lstinline{"}' key.  The `6' quote
marks are produced by a different key to the `9' quote marks.  See
\autoref{sec:quotations}.

\section{`Fragile' commands in `moving' arguments}

\subsection{Problem}

Some arguments to \LaTeXe{} commands are known as \emph{moving
  arguments}.  These are arguments that are potentially typeset
elsewhere in the document (and may or may not be typeset at the point
they occur).  An example is the title of a chapter; as well as being
typeset at the point it occurs it will be typeset a second time as
part of the table of contents.

Some \LaTeXe{} commands are \emph{fragile}: they break if moved.
These are few in number, and they are rarely used in moving arguments
(this document contains none).  Examples include
\lstinline|\footnote{|\ldots\lstinline|}|, \lstinline|\begin|,
  \lstinline|\end| and all commands with an optional argument.
\begin{lstlisting}
  \chapter{Typesetting footnotes\footnote{Footnotes are fragile}}
\end{lstlisting}
causes an error.

\subsection{Solution}
Fragile commands in moving arguments should be \emph{protected}.
\begin{lstlisting}
  \chapter{Typesetting footnotes\protect\footnote{Footnotes are fragile}}
\end{lstlisting}
does not break.

That example would give an odd-looking table of contents.  Even better
is:
\begin{lstlisting}
  \chapter[Typesetting footnotes]{%
    Typesetting footnotes\footnote{Footnotes are fragile}}
\end{lstlisting}
which moves the optional argument and typesets the compulsory argument
in place.  (See also \autoref{lst:label}.)

\section{\BibTeX{} gotchas}
\label{sec:BibTeXgotchas}

There are two features that often catch people out when preparing
bibliography files.

\begin{enumerate}
\item Multiple authors in an author field must be separated with
  `\lstinline|and|':
  \begin{lstlisting}[gobble=4]
    author = {John Smith and Brown, Mary and Joe Green and Lillian White}
  \end{lstlisting}
  Commas must \emph{not} be used to separate names as they are used to
  indicate a surname occurring before the forename (as in `Brown,
  Mary').
\item \BibTeX{} may change capitalisation of your text.  Capital
  letters that must stay as capital letters should be protected from
  \BibTeX's formatting by braces:
  \begin{lstlisting}[gobble=4]
    title = {A Guide to {C++}: Its use with {Z}, {B} and {Alloy}}
  \end{lstlisting}
\end{enumerate}

\cleardoublepage
\chapter{Running \LaTeX{} on the departmental systems}
\label{cha:deptfac}

\section{Under GNU/Linux}
\label{sec:GNU/Linux}

The default departmental \TeX{}-and-friends installation is the latest
\TeX{}Live distribution (see
\url{http://www.cs.york.ac.uk/support/texlive.php}).  You will need
\lstinline|/usr/local/bin| in your \lstinline|PATH| to access it.

You may also wish to set a variable called \lstinline|TEXINPUTS| if
you have private collections of macros.  This variable is a
colon-separated list of directories (just like \lstinline|PATH|) for
\TeX{} to search.  If you keep everything in the same directory then
the default value should be good enough.  The default value is
\lstinline|.::|, the empty directory name meaning `the standard
library installed with \TeX{}Live'.

To run \LaTeXe{} from the command line on a file called
\lstinline|foo.tex| you type the command `\lstinline|pdflatex foo|' to
produce a PDF file called \lstinline|foo.pdf|.  Similarly, to extract
the bibliographic information associated with \lstinline|foo.tex| you
should run `\lstinline|bibtex foo|' \emph{after} running \LaTeXe{} ---
see \autoref{cha:editprocess}.

If you want to use AUC\TeX{} with emacs then you need to put
\begin{lstlisting}[language=lisp,gobble=2]
  (load "auctex.el" nil t t)
  (load "preview-latex.el" nil t t)
\end{lstlisting}
in your \lstinline|.emacs| file.


\section{Under Microsoft Windows}
\label{sec:MSW}

I have never done this.  I am told that \lstinline|WinEDT| is the tool
to use.

\cleardoublepage

\chapter{\autoref{lst:quoteex} typeset}
\label{cha:quoteex}

The typeset version of \autoref{lst:quoteex} is between the horizontal
lines.  Note that \TeX{} could not find an ideal break for the 3rd
paragraph (unsurprisingly, as this is a difficult text to typeset),
and so has let one line protrude too far.

\noindent\rule{\textwidth}{1pt}
\input{joyce}
\noindent\rule{\textwidth}{1pt}

\end{document}
